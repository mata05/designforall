%
\documentclass[parskip,headsepline, headtopline, %
footsepline, oneside, 12pt, headings=small]{scrreprt}
\usepackage[ngerman]{babel}
\usepackage[utf8]{inputenc}
\usepackage{color}
\usepackage{pdfpages} % To include other PDFs
\usepackage{changepage} 
\usepackage{textcomp}
% To insert dummy text
\usepackage{blindtext}
\setcounter{secnumdepth}{2}
 
% Provide a command to include pretty quotes
\usepackage{ragged2e} %justify dictum
\renewcommand*{\dictumwidth}{.5\textwidth}
\newcommand{\setChapterQuote}[3]{\setchapterpreamble[o]{%
\dictum[#2 \emph{#3}]{\justifying {#1}}}}
 
%Set font to times
\usepackage{txfonts}
 
% Define own Chapter style
% Pretty chapter pages
%------------------------------------------
\definecolor{nicered}{rgb}{.647,.129,.149}
\definecolor{darkblue}{rgb}{0,0,.5}
\usepackage{soul}

% hyperref has to be used last
\usepackage[
    bookmarks,
    bookmarksopen=true,
    colorlinks=true,
% diese Farbdefinitionen zeichnen Links im PDF farblich aus
    linkcolor=darkblue, % einfache interne Verkn�pfungen
    anchorcolor=darkblue,% Ankertext
    citecolor=darkblue, % Verweise auf Literaturverzeichniseintr�ge im Text
    filecolor=darkblue, % Verkn�pfungen, die lokale Dateien �ffnen
    menucolor=darkblue, % Acrobat-Men�punkte
    urlcolor=darkblue, 
% diese Farbdefinitionen sollten f�r den Druck verwendet werden (alles schwarz)
    %linkcolor=black, % einfache interne Verkn�pfungen
    %anchorcolor=black, % Ankertext
    %citecolor=black, % Verweise auf Literaturverzeichniseintr�ge im Text
    %filecolor=black, % Verkn�pfungen, die lokale Dateien �ffnen
    %menucolor=black, % Acrobat-Men�punkte
    %urlcolor=black, 
    %backref,
    plainpages=false, % zur korrekten Erstellung der Bookmarks
    pdfpagelabels, % zur korrekten Erstellung der Bookmarks
    hypertexnames=false, % zur korrekten Erstellung der Bookmarks
    linktocpage % Seitenzahlen anstatt Text im Inhaltsverzeichnis verlinken
]{hyperref}

\makeatletter
\newsavebox{\feline@chapter}
\newcommand\feline@chapter@marker[1][4cm]{%
\sbox\feline@chapter{%
\resizebox{!}{#1}{\fboxsep=1pt%
\colorbox{grey}{\color{white}\bfseries\sffamily\thechapter}%
}}%
\rotatebox{90}{%
\resizebox{%
\heightof{\usebox{\feline@chapter}}+\depthof{\usebox{\feline@chapter}}}%
{!}{\scshape\so\@chapapp}}\quad%
\raisebox{\depthof{\usebox{\feline@chapter}}}{\usebox{\feline@chapter}}%
}
\newcommand\feline@chm[1][4cm]{%
\sbox\feline@chapter{\feline@chapter@marker[#1]}%
\makebox[0pt][l]{% aka \rlap
\makebox[1cm][r]{\usebox\feline@chapter}%
}}   
 
\renewcommand*{\chapterformat}{%
\hspace{\leftmargin} \feline@chm[2.5cm] % Height of the colored box
\hspace{2cm}
}
\makeatother
%------------------------------------------
% ------------------------------------------------------------------------------
\newcommand{\HRule}[1]{\hfill \rule{0.2\linewidth}{#1}}         % Horizontal rule

\definecolor{grey}{rgb}{0.9,0.9,0.9} 

\makeatletter                                                   % Title
\def\printtitle{%                                               
    {\centering \@title\par}}
\makeatother                                                                    

\makeatletter                                                   % Author
\def\printauthor{%                                      
    {\centering \large \@author}}                               
\makeatother                                                    

% ------------------------------------------------------------------------------
% Metadata (Change this)
% ------------------------------------------------------------------------------
\title{ \fontsize{50}{60}\selectfont \vspace{2.10cm}
\hfill \begin{huge}{\fontfamily{@arialn}\selectfont {\fontfamily{txtt}\selectfont The Design for All}}\end{huge}
 \hfill \large{\begin{flushright}Psychologie, Design, Emotionen\end{flushright}} \vspace{1.9cm}
 \hfill \small{Design for All \textbackslash\textbackslash Prof. Dr. Susanne Maaß \textbackslash\textbackslash  Wintersemester 2012/2013 \textbackslash\textbackslash  \today} 
}
\author{
                \hfill Martha Rohte (2259987) und Maria Meister (2142018)\\  
                \hfill Universität Bremen\\   
                \hfill Fachbereich Mathematik \& Informatik \\
        \hfill \texttt{\fontfamily{cmr}\selectfont mata@informatik.uni-bremen.de} \\
        \hfill \texttt{\fontfamily{cmr}\selectfont mmeister@informatik.uni-bremen.de} \\
}


\renewcommand{\sfdefault}{@sshlined}
\begin{document}

% ------------------------------------------------------------------------------
% Maketitle
% ------------------------------------------------------------------------------
\thispagestyle{empty}                           % Remove page numbering on this page



\colorbox{grey}{
        \parbox[t]{1.14\linewidth}{
                \printtitle 
                \vspace*{0.2cm}               
        }
}
        \vfill
\printauthor                                                            % Print the author data as defined above
\HRule{1pt}

\clearpage

%\begin{abstract}

%\end{abstract}
% NOTE: '?' is placeholder for selected text 
% or caret location (when no selection).

%Die Ausarbeitung sollte mindestens folgende Teile haben
%• Titelblatt mit Angaben zum Thema, VerfasserInnen mit Matrikelnummern,
%Veranstalterin, Veranstaltung, Semester, Datum...
%• Gliederung/ Inhaltsverzeichnis
%• eine Einleitung, die auch die Struktur des Papiers beschreibt,
%• formulierte Übergänge zwischen den Abschnitten,
%• eine Zusammenfassung am Schluss und
%• ein vollständiges Verzeichnis der verwendeten Literatur/Quellen
%• Seitenzahlen!
%Umfang: pro Person in der Vorbereitungsgruppe erwarte ich 8-10 Seiten.
%Schriftgröße Arial 11 oder Times New Roman 12. Seitenränder 2,5 cm

\tableofcontents
 \clearpage
% Chapter 0

\chapter*{Einleitung}
\addcontentsline{toc}{chapter}{Einleitung}


In Rahmen der Veranstaltung \textbf{Design for All} haben wir uns mit dem Buch The Design of Everyday Things auseinandergesetzt. Dieses befasst sich mit der  Benutzbarkeit von Alltagsgegenständen bzw. deren Nichtbenutzbarkeit durch falsches Design. In dieser Ausarbeitung wollen wir kurz die Problematik aufführen, um uns dann der Frage zu widmen, woher schlechtes Design kommt bzw. was gutes Design ausmacht. Im Anschluss dazu werden wir den Bezug auf die Veranstaltung nehmen, um eine umfassendes Resümee ziehen zu können.

\begin{figure}
\center
\includegraphics[width=.4\textwidth]{images/HTC-Butterfly}
\includegraphics[width=.4\textwidth]{images/htc-mini-fernbedienung}
\caption{Das HTC Butterfly und die dazugehörige Fernbedienung}
\label{fig:htc}
\end{figure}

Anfang diese Jahres veröffentliche die taiwanesische Firma HTC das HTC Mini \cite{htc}, wobei es sich um eine Fernbedienung für das HTC Butterfly Handy handelt. Mit Hilfe dieser Fernbedienung werden allerlei Funktionen ausgelagert. Grundsätzlich könnte man sich nun fragen, warum ist es überhaupt notwendig ist eine Fernbedienung für ein Mobiltelefon zu haben? Denkbar wären da mehrere Gründe, zum einen zu viel Funktionalität für den verfügbare Fläche, da die Displaygröße eines der ausschlaggebendsten Kriterien für den Kauf eines Mobiltelefons ist. Zum andern können die physischen Knöpfe als Bedienungselemente nicht nur als bekannt voraussetzen werden, sondern auch auf haptischer Ebene zu größerer Befriedigung führen. Durch die Auslagerung wird aber auch der Funktionsumfang auf den jeweiligen Geraten verringert.
Grundsätzlich scheint der Umfang der Funktionalität zu komplex zu sein, um sie auf ein Produkt abzubilden. Eine Problematik mit der heutzutage viele Produktdesigner konfrontiert sind. es stellt sich nun auch auch die Frage, ob man einen großen Fokus auf die Benutzbarkeit oder das Design eines Produktes legt. Meistens wird in dieser Frage aus ökonomischen Gründen zu Gunsten des Design bzw. der Ästhetik entschieden.
Mit dieser Diskrepanz setzt sich Norman in seinen Buch auseinander, dort zeigt er die Folgen und  seine selbst entwickelten Gegenmaßnahmen auf, um trotzdem ein gewissen Grad an Benutzbarkeit sicherzustellen.

% Chapter A
\setChapterQuote{A thought is an idea in transit.}{Pythagoras}{582 B.C. - 497 B.C.}
\chapter{Alltagsdesign}


Der das Buch geschrieben hat, ursprünglich Psychologe, setzt er sich immer mehr mit der Benutzbarkeit zuerst von Alltagsgegenständen und auch von informatischen Produkten auseinander. Sein Buch \glqq The Design of Everyday Things \grqq ist zu großer Beliebtheit in den Kreisen der Usabilityforschung und des Designs gelangt. 


Norman sieht den Ursprung von schlechten Design hauptsächlich in einen Phänomen, welches er in seinem Buch als Technologieparadoxon wie folgt beschreibt \cite[S.31]{don}:\\

\glqq Whenever the number of functions and required operations exceeds the number of controls, the design becomes arbitrary, unnatural, and complicated. 
	The same technology that simplifies life by providing more functions in each device also complicates life by making the device harder to learn, harder to use. This is the paradox of technology.\grqq 

Oder anders ausgedrückt Systeme werden immer komplexer, da sie über im mehr Funktionalitäten verfügen, allerdings können diese aufgrund ihrer Anzahl nicht eindeutig abgebildet werden, was dazu führt das Funktionen oft nur durch Kombination anderer Funktionen ausgeführt werden können. Durch Umstand erhöht sich die Komplexität von Systemen immens. Zusätzlich dazu bekommt der Benutzer meist nur wenig Feedback, ob seine Interaktion mit dem System erfolgreich war. darauf wird gerne zu Gunsten des Designs verzichtet. 


\section{Designaspekte}
\label{sec:aspekte}

Laut Norman ist das zentrale Prinzip von Designen, Dinge so zu entwerfen, dass die Erwartungshaltung des Nutzers bezogen auf die Funktionsweise der einzelnen Elemente auch mit ihrer tatsächlichen Funktionalität übereinstimmt, sodass der Nutzer in der Lage versetzt wird akkurat agieren zu können.\\ 
Als Ausgangsbedingung dafür gibt es fünf zentrale Aspekte, die eingehalten werden müssen, um  ein möglichst intuitives Design bereitzustellen. Die Prinzipien 
	
Unter den Aspekt der \textbf{Sichtbarkeit} versteht Norman Funktionalitäten transparent zu gestalten. Beispielsweise muss bei öffentlichen Waschbecken ersichtlich gemacht werden, ob diese manuell durch Knöpfe oder über Sensoren reguliert werden.\\
   
Des Weiteren ist ein unabgänglicher Aspekt, der des \textbf{Feedback}s. Darunter versteht man jegliche Rückmeldung des Systems, über Auswirkungen von Aktionen, sowie der daraus folgende interne Zustand des Systems. Das Feedback kann auf visueller (z.B Signalleuchten) oder auch auf akustischer Ebene erfolgen.\\
 
Das	\textbf{Mapping} bezieht sich darauf die inhärenten Funktionen des Systems auf einer Art und Weise abzubilden, sodass sie nicht nur eindeutig identifizierbar sind, sondern auch intuitiv verständlich.\\

Der Aspekt der \textbf{Affordanz} basiert auf den Affordanzbegriff, der 1977 im Artikel 'The Theory of Affordances' von James Jerome Gibson eingeführt wurde. Der Begriff Affordanz leitet sich vom Englischen “to afford”(“leisten”, “anbieten”, “gewähren”) ab. Laut Gibson leiten sich Affordanzen aus Komplementarität von Umwelt und Lebewesen ab, d.h.  
die Handlungsanregung von Dinge aufgrund der Informationen über ihre funktionell relevante Eigenschaften oder Bestandteilen der Umwelt. Dieses Zusammenspiel ermöglicht oder viel mehr suggeriert ein bestimmtes Verhalten. Wodurch die Affordanz maßgeblich für Funktionalität und Intuitivität eines Gegenstandes wird.
Norman erweiterte diese Definitionsbegriff um den der \textit{wahrgenommenen} Affordanz. Darunter fallen auch suggerierte Affordanzen, die gar nicht tatsächlich bestehen, wie beispielsweise aufgemalte Türen durch die man nicht gehen kann oder der Computermonitor, der trotz seiner Begebenheiten nicht haptisch benutzbar ist. 

\begin{figure}
\includegraphics{images/figure2_affordances.png}
\caption{Affordanzen}
\label{fig:affordanz}
\end{figure}

Der Designaspekt Constraints beschreibt Beschränkungen, die aufgrund bestimmter Begebenheiten entstehen. Die Begebenheiten können von \textit{physisch}, \textit{semantisch}, \textit{kulturell} oder \textit{logische} Art sein.

\begin{enumerate}
\item Bei der \textbf{physische Constraints} , sind eng mit den Affordanzen eines Gegenstandes verknüpft. es ist nicht möglich über den Affordanzraum hinaus zu handeln. Nicht jeder Schlüssel passt ins Schloss.
\item Die \textbf{semantische Constraints} folgen allgemein hin gültigen Regeln, wie zum Beispiel,  das man Legosteine nur von oben oder unten zusammenbauen kann und nicht etwa von der Seite. Oder man sich beim Autofahren  nicht mit den Rücken zum Lenkrad setzen kann.
\item Die \textbf{kulturelle Constraints} , dabei handelt es sich um Einschränkungen, die sich durch den kulturellen Hintergrund einer Person oder auch Gruppe ergeben. Sie sind somit gesellschaftlich angelernt.
\item Die \textbf{logische Constraints} sind aufgrund bestimmter vorliegender Prämissen (können auch andere Affordanzen sein) geben, oder in anderen Worten per \glqq Ausschlussverfahren\grqq. Wenn drei Antworten existieren und a und b falsch sind muss infolgedessen c die richtige Antwort sein. 
\end{enumerate}


Ein von Norman angeführtes Beispiel war ein Experiment, indem Probanden einen Polizisten aus  Lego modellieren sollten (Vgl. Abbildung).
Auf der elementarsten Ebene wären die physischen Constraints, da nicht alle Legos gleich gut ineinander passen. 
Die  Polizistenfigur im Modellbausatz muss auf den Motorrad vorwärts sitzen. Der kulturelle Constraint würde in diesem Fall vorgeben ob eine Person die Legofigur auf die linke oder rechte Seite der Straße stellen würde.  Ein logischer Constraint in diesem Beispiel wäre, wenn ich zwei Lichter habe (rot, weiß) und das weiße als Frontlicht benutze, muss ich das andere Licht als Rücklicht verwenden. 

\section{Konzeptuelle Modelle}

Das Konzeptuelle Modell eines Systems unterliegt dem gedanklichen Modell (Designmodell) des Designers, welches er als als System abbildet. Das gedankliche Modell des Benutzers wird durch sein Vorwissen und das Systembild gebildet. Unter Systembild versteht Norman die sichtbaren Teile des Systems auf Grund deren sich ein Benutzer die Funktionalität erschließt (Hebel , Knöpfe etc.). Probleme können nun dadurch geschehen, dass der Designer sein Designmodell unzureichend auf das System abbildet, sodass sich für den Benutzer ein falsches Systemmodell darlegt, auf Grund dessen er sich ein falsches Modell bildet. Oder auch wenn der Designer ein Modell wählt, welches nicht zwingend dem des Benutzers entspricht.

Daraus leiten sich für Norman die  zwei wichtigsten Regeln benutzerfreundlichen Designs ab. Zum einen, dass es essentiell ist für den Gebrauchsgegenstand ein gutes konzeptionelles Modell zu wählen und zum anderen, dass wichtige Teile sichtbar gemacht werden müssen.

\section{Aktion}

Nach Norman laufen die Aktionen in einem Zyklus ab (Vgl. Abb. \ref{fig:action}, \cite[S. 46ff]{don}). Dieser besteht aus zwei Teilen: dem Gulf of Execution und dem Gulf of Evaluation. Zuerst muss der Handelnde in seinem Kopf ein Ziel bilden wie zum Beispiel \glqq Ich möchte lesen (können, aber dazu ist es zu dunkel).\grqq Um eine Handlung auszuführen, muss er nun den Gulf of Execution überbrücken. Hierzu muss er zuerst die Intention bilden \glqq Ich schalte die Leselampe an.\grqq, die Aktion spezifizieren \glqq Ich muss den Arm ausstrecken, den Schalter fassen und den Schalter umlegen.\grqq und die Aktion schließlich auch wirklich ausführen. Als Resultat ändert sich der Zustand der Welt. Um diesen wahrzunehmen und zu bewerten, muss der Handende nun den Gulf of Evaluation überbrücken. Dies beginnt damit, dass der Zustand der Welt wahrgenommen wird \glqq Es ist (nicht) hell.\grqq Danach wird der Zustand der Welt interpretiert: \glqq Die Lampe scheint (nicht).\grqq Letztlich wird das Ergebnis evaluiert: \glqq Ich kann jetzt (immer noch nicht) lesen.\grqq

\begin{figure}
\includegraphics[width=\textwidth]{images/ActionCycle.png}
\caption{Der Aktionszyklus von Norman}
\label{fig:action}
\end{figure}

Damit ist dieser Zyklus beendet. Ist das Ergebnis positiv, so wird ein neuer Zyklus begonnen oder ein übergeordneter Zyklus fortgesetzt (\glqq Ich lese das Buch. / Ich fange an das Buch zu lesen. \grqq). Ist das Ergebnis hingegen negativ, so wird der Zyklus entweder mit einer anderen Intention wiederholt (\glqq Ich will Licht. \textrightarrow Ich schalte die Zimmerlampe an. \grqq) oder der Zyklus wird verfeinert (\glqq Ich überprüfe die Funktionsfähigkeit der Lampe: Hat sie Strom? Ist die Glühlampe defekt?  \grqq). Die verschiedenen Aktionszyklen können beliebig ineinander übergehen bzw. verschachtelt werden. 

Ein gutes Design verkleinert dabei die beiden Gulfs soweit wie möglich, denn je geringer die Schritte sind, desto unwahrscheinlicher ist es Fehler zu machen und umso weniger Zeit vergeht eine Aktion auszuführen. Für die Ausführung sind dabei die Sichtbarkeit, das Mapping, die Affordanzen und die Constraints entscheidend; all dies kann bei guter Anwendung die Überquerung des Gulf of Execution vereinfachen. Hingegen ist für die Evaluation das Feedback am wichtigsten, denn dieses wird an der Veränderung der Welt gemessen. Je direkter und einfacher zu interpretieren das Feedback ist, umso schmaler ist der Gulf of Evaluation. Ein gutes Beispiel hierfür sind die altbekannten Lampen: sobald man den Schalter gedrückt hat geht die Lampe an --- oder eben auch nicht. Wobei bei neueren Lampen, so wie zum Beispiel Energiesparlampen, das Feedback zwar noch ausreichend aber nicht mehr ganz so überragend ist: die Lampen leuchten oft nur verzögert auf oder werden erst kontinuierlich hell, im ersten Moment mag der uneingeweihte Benutzer vermuten, dass die Lampe defekt ist. 
%Bsp. einfaches aber aufwändigeres Feedback ist Flicken vom Fahrradschlauch: Ist er dicht?

\section{Fehler}

Da irren nun einmal menschlich ist, unterlaufen allen Menschen gelegentlich Fehler. Norman unterscheidet diese Fehler in zwei grundlegende Arten \cite[S. 105]{don}: Die Erste sind die Fehlleistungen, im Englischen \glqq Slips\grqq  genannt. Diese beschreiben unintendierte Handlungen. Die Zweite sind die Irrtümer, im Englischen \glqq Mistakes\grqq. Sie beschreiben nicht erwartete und erwünschte Ergebnisse, die auf vermeintlich korrekte Handlungen folgen.

\subsection{Fehlleistungen (Slips)}

Die Fehlleistungen beruhen im Allgemeinen darauf, dass eine Handlung automatisiert abläuft oder unaufmerksam auf Grund ihrer (vermeintlichen) Einfachheit ausgeführt wird. Sie sind meist einfach zu erkennen und sollten in einem guten Design einfach zu berichtigen sein.

Norman unterscheidet sechs verschiedene Arten von Fehlleistungen\cite[S. 107]{don}:

\begin{itemize}
\item \textbf{Fangfehler} sind Fehler, die passieren, wenn die Anfangssequenz von zwei Aktionen identisch sind, wobei die beabsichtigte Aktion unvertrauter ist und daher von der vertrauteren \glqq aufgefangen\grqq wird. Zum Beispiel, wenn man sein Geburtsdatum schreiben soll, aber als Jahr das aktuelle Jahr einträgt.
\item \textbf{Beschreibungsfehler} erfolgen, wenn zwei Aktionen viel Ähnlichkeit miteinander haben. Oft wird nur ein Gegenstand dabei ausgewechselt, also zum Beispiel wird die Dreckwäsche nicht in den Wäschekorb, sondern in die Toilette wirft, die direkt daneben steht: Aufklappen -- rein werfen -- zuklappen.
\item \textbf{Datengesteuerte Fehler} kommen vor, wenn man zwei verschiedene Daten miteinander vertauscht, vor allem, wenn man das falsche Datum in dem Moment sieht, wenn man also zum Beispiel statt der Telefonnummer die Raumnummer wählt.
\item \textbf{Fehler durch assoziative Aktivierung} finden dann statt, wenn die Impulse zur Aktivierung dieser Aktionen ähnlich sind und deshalb die falsche Handlung ausgeführt wird, zum Beispiel, wenn man in das Telefon \glqq Herein!\grqq  ruft.
\item \textbf{Fehler durch Aktivierungsverlust} geschehen aufgrund einer Aktion, die während einer anderen Handlung ausgeführt wird, diese vergessen wird. Das typische Beispiel ist, dass man nachdem man die Tür zu einem Raum geöffnet hat und in diesem steht man nicht mehr weiß, was man dort tun wollte.
\item \textbf{Modus-Fehler} treten ausschließlich auf, wenn Geräte verschiedene Modi haben. Wird versucht eine Handlung auszuführen, die in einem Modus nicht möglich ist, obwohl sich das Gerät in diesem befindet, so ist dies ein Modus-Fehler. Ein Beispiel ist, wenn man versucht in einem Dokument zu schreiben während dieses im schreibgeschützten (read-only) Modus geöffnet ist.
\end{itemize}

Nach Möglichkeit sollte während des Designvorgangs darauf geachtet werden, dass diese Fehlleistungen vermieden werden, dass das Design also möglichst Fehlertolerant ist. Aus diesem Grund sind zum Beispiel in einem Firmennetzwerk häufig die Telefonnummern mit den Raumnummern identisch.

\subsection{Irrtümer (Mistakes)}

Irrtümer sind im Gegensatz zu Fehlleistungen oft nur schwer oder spät zu erkennen und deshalb auch nur schwer zu vermeiden. Hinzu kommt, dass sie oft weitaus drastischere Folgen als Fehlleistungen haben. Sie sind das Resultat von dem Verfolgen ungeeigneter Ziele \cite[S. 114]{don}. Ihr Ursprung liegt viel mehr in dem Entscheidungsverhalten der Menschen, darin, dass Erfahrungswerte oft höher bewertet werden als logisches Denken, beziehungsweise letzteres noch nicht einmal bemüht wird, wenn ersteres vorliegt.

\section{Wissen}

Die Funktionsweise von allen Dingen, die komplexer sind als ein gewöhnlicher Hocker, muss der Benutzer auf irgendeine Weise erlernt haben. Dies kann grundsätzlich auf zwei verschiedene Arten geschehen: Zum Einen kann der Benutzer das Wissen über die Funktionsweise auswendig gelernt haben; das Wissen befindet sich also im Kopf. Zum Anderen kann sich das Wissen auch direkt aus dem Design erschließen (siehe dazu Kapitel \ref{sec:aspekte}), das Wissen ist also unmittelbar bei jeder Benutzung verfügbar \cite[S. 54ff]{don}. 
Dabei bestehen zwischen beiden Möglichkeiten mehrere Trade-offs:

\begin{tabular}{p{.4\textwidth}p{.4\textwidth}}\vspace{6pt}
\hspace{1cm}\textbf{Wissen im Kopf} & \vspace{6pt}\hspace{1cm}\textbf{Verfügbares Wissen}\\
\begin{itemize}
\item Großer Lernaufwand
\item Langzeitgedächtnis
\item Hohe Effizienz
\item Erste Verwendung schwer
\item Unabhängig von der Umgebung
\end{itemize} &
\begin{itemize}
\item Geringer/kein Lernaufwand
\item Kurzzeitgedächtnis
\item Geringe Effizienz
\item Erste Verwendung leicht
\item Abhängig von der Umgebung
\end{itemize}
\\
\end{tabular}

Welche Art nun gewählt wird, hängt stark von dem Anwendungskontext ab, wobei im Regelfall eine Mischung von beidem vorliegt, so dass der erfahrene Benutzer nicht mehr nachschauen muss, wie es funktioniert. Hierfür ist die Schreibmaschinentastatur ein perfektes Beispiel: Benutzer, die das zehn-Finger-System beherrschen, brauchen nicht mehr auf die Tastatur zu schauen um jeden einzelnen Buchstaben zu finden, womit sich ein unerfahrener Benutzer aber durchaus noch behelfen kann, auch wenn er für die gleiche Menge an Text deutlich länger brauchen wird.\\
Ein anderes Beispiel sind sicherheitskritische Anwendungen. Gewöhnlich werden diese, wie beispielsweise die Bedienung eines Flugzeuges, mit viel Aufwand und Kosten erlernt und immer wieder geübt, damit sie im Ernstfall zumindest halbautomatisch ablaufen und darüber hinaus weitaus weniger Zeit beanspruchen als wenn der Benutzer erst herausfinden muss, wie es bedient wird.\\
Im Gegensatz dazu sollten Anwendungen, die jeder benutzen können soll und dies aber nur selten oder nicht regelmäßig tut, auf den ersten Blick verständlich sein. Hierfür ist ein Fahrkartenautomat der Deutschen Bahn ein gutes Beispiel: Der durchschnittliche Benutzer ist ungeduldig, will keine große Suche betreiben, nicht irritiert werden und hat meistens eine kleine Phobie vor der Bedienung dieser Automaten, die umso größer ist, je seltener die Automaten benutzt werden. Da vermutlich niemand die Funktionsweise auswendig lernen möchte, muss sie also aus dem Automaten selbst ersichtlich sein. Auch wenn es seit ein paar Jahren eine neue Software für diese Automaten gibt, die sowohl die Bedienbarkeit als auch das Aussehen stark verbessert haben, so scheinen noch genug Leute diese nicht zu verstehen, dass eine extra Anwendung --- über die Internetseite der Deutschen Bahn zugänglich --- existiert, mit der man die Bedienung dieser Automaten erlernen kann \cite{bahn}. Wobei man sowohl mit der alten als auch mit der neuen Software durch alle Schritte geleitet wurde und man problemlos an seine Fahrkarte kam --- wenn man genug Geduld aufbrachte, um die Anweisungen genau durchzulesen.


 
% Chapter B
\setChapterQuote{Much wisdom often goes with fewest words.}{Sophocles}{496 B.C. - 406 B.C.}
\chapter{Design und Emotion: Von Usability zu User Experience}

In den letzten Jahren kam es im Bereich des Interaktionsdesigns zu einer Fokusverschiebung von Usability zu User Experience. Dies beruht auf weiterführend gewonnenen Erkenntnissen in der Psychologie, insbesondere in der Kognition. Zeitgleich dazu häufte sich die vielfache Kritik von Designern als Resonanz auf Normans Buch die bemängelten, dass bei Einhaltung seiner Prinzipien die Produkte zwar benutzbar wären, aber gleichfalls auch hässlich aussähen \cite[S. 8]{don2}. Angespornt durch diese Kritik widmete er sich der Frage, inwiefern unsere Wahrnehmung von Dingen durch unsere Emotionen beeinflusst wird.

Beispielsweise
Daraus leitete er die Theorie der drei Ebenen (three levels of design \cite[siehe Kapitel 3]{don2}) ab: \textit{Visceral} (Viszeral), \textit{Behavioral} (Verhalten) und \textit{Reflective} (Reflektierend). Diese unterschiedlichen Verarbeitungsebenen des Gehirns lassen sich wie folgt beschreiben:
\begin{enumerate}
\item Bei der \textbf{viszeralen Ebene} handelt es sich um die Ästhetik des Designs
\item Die \textbf{Verhaltensebene} beschreibt das Verhalten bezüglich der Usability eines Gegenstandes
\item Die \textbf{reflektierende Ebene} schlussendlich reflektiert die Beziehung zum Gegenstand, die durch diesen beschrieben wird
\end{enumerate}

\begin{figure}
\center
\includegraphics[width=.7\textwidth]{images/ipod-classic}
\caption{Der iPod classic \cite{ipod}}
\label{fig:ipod}
\end{figure}

% Nähere Bescheibung der Ebenen (am Beispiel)

Der iPod classic (siehe Abb. \ref{fig:ipod}) sieht apart aus, fühlt sich in der Hand toll an und die Abgerundetheit ist für das Auge angenehm. All dies macht die viszerale Ebene aus und der iPod brilliert darin.

Auch die Verhaltensebene ist bei dem iPod sehr gut eingebracht: Die wichtigen Bedienungen hat man sofort, mit wenig Aufwand wird genau das Lied abgespielt, das gerade gewünscht ist, und die Batterie hält sehr lange. Es ist also eine Wonne, ihn zu benutzen.

Auch auf der reflektierenden Ebene stellt der iPod seine Qualität zur Schau: Er ist zu einem Statussymbol geworden, das einen gewissen Wohlstand und auch Trendbewusstsein widerspiegelt. 


Dabei muss jedem Level eine besondere Beachtung geschenkt werden, da es eine Arbeitsweise von Benutzern widerspiegelt und dementsprechend einen eigenen Designstil erfordert. Zu beachten ist dabei, dass sich die Prinzipien der einzelnen Ebenen teilweise gegenseitig widersprechen. Dem zu Grunde liegend ist es nur allzu verständlich, dass nicht alle Ebenen kompromisslos abgedeckt werden können. 
Zudem kommt, dass bei unterschiedlichen Gefühlslagen verschiedene Arten von Denkprozessen stattfinden. Gemäß dieser Aussage verwendet man beispielsweise bei einer positiven Gefühlslage Denkstrukturen, die einer Breitensuche gleichen, man denkt also kreativer und mehr out-of-the-box. Analog dazu gleichen die Denkstrukturen bei einer negativen Gefühlslage denen einer Tiefensuche, man ist stark fokussiert und detailbewusster.

Aus der drei Ebenen Theorie und der Tatsache, dass die emotionale Lage eines Benutzers seine Denkstrukturen maßgeblich vorgibt, lässt sich zum Einen ableiten wie beziehungsweise auf welchen Ebenen der Benutzer die Gegenstände wahrnimmt, zum Anderen, dass die Gegenstände ebenfalls den Benutzer beeinflussen. Somit wäre gewissermaßen eine Symbiose zwischen Benutzer und Gegenständen auf emotionaler Ebene hergestellt. Unter Berücksichtigung dieser Wechselbeziehung muss bei den vorgestellten Designprinzipien von Norman ein größerer Fokus auf die Ästhetik und die ideellen Werte gelegt werden.

 
% Chapter C
\chapter{Resümee}
\section{Resümee}

Aus der Problematik heraus, dass Systeme zunehmend komplexer werden und diese Funktionalitäten nur beschränkt sichtbar auf den Gerät gemacht werden können, erhöht sich die Schwierigkeit in der Benutzbarkeit für den User. Zusätzlich werden viele Funktionalitäten auch auf Grund der Ästhetik versteckt, was die  Benutzbarkeit noch weiter erschwert. Auch wenn Benutzbarkeit oft vernachlässigt wird, ist sie schlussendlich essentiell für den Umgang mit Gegenständen und definiert nicht nur die Nutzbarkeit sondern auch die Freude daran es zu benutzen. Freude macht alles was leicht (intuitiv) zu benutzen ist und/oder Spaß macht es zu benutzen. \\
Wenn man dieser Beobachtung Beachtung schenken will, muss man sich darüber in klaren sein, das verschiedene Parteien von Menschen in den Designprozess  eines Systems involviert sind, deren Verhalten wiederum durch unterschiedliche Faktoren motiviert ist. Darüber hinaus wird ist System selbst durch über seine inhärente Eigenschaften (Sichtbarkeit, Affordanz, Constraints) definiert. Die kognitive Vorgänge der involvierten Personen(Wahrnehmung, Fehler, Wissen) bestimmen ihr jeweilige Vorstellung von dem System wir auch die Benutzungsweise des Systems. \\ 
Auf der einen Seite gibt es eine Beziehung zwischen Designer und System. Der Designer sich ein Designmodell kreiert und dieses als System abbildet. Wobei er sich über seine Benutzergruppe im Klaren sein sollte und in ihren Interesse entwerfen und nicht etwa nur aus seiner persönlichen Sicht heraus.
Die sichtbaren Teile eines solchen Systems definieren das Systemmodell. Der Benutzer wiederum erstellt sich basierend auf seiner bisheriger Erfahrung und den Systemmodell das Benutzermodell.\\
Um zu gewährleisten das Benutzermodell mit dem Designermodell übereinstimmt oder gegebenenfalls angepasst wird, sollte ein gutes Mapping und genügend Feedback vorhanden sein. Ein richtiges Benutzermodell sorgt dafür, dass der Benutzer seine notwendigen Aktionen intuitiv richtig ableiten, dies erhöht nicht nur die Benutzbarkeit nicht nur auf funktionaler ebene sondern auch den Spaßfaktor (Userexperience) allgemein. \\ 
Generell haben Emotionen einen weitaus größeren Einfluss auf das Verhalten der Menschen als ursprünglich angenommen. Die zunehmender Wichtigkeit von Userexperience beim dem Designen von Systemen sorgt dafür, dass immer mehr Modelle im Fachgebiet Design entstehen, die diese berücksichtigen.
Eins davon ist das vorgestellte drei Ebenen Modell nach Norman. Durch dieses wird auch die Ästhetik in den Designprozess mit einbezogen und die Benutzbarkeit, die Norman in seinem ersten Buch so hervorhebt, ist in diesem Modell nur ein Aspekt von dreien. \\
Die oben aufgezählten Faktoren sind allerdings kaum oder nur schwer in einen Softwaredesignprozess zu integrieren, da sie meist kognitive Natur sind und somit nahe zu unmöglich formalisierbar. Zudem  wird von den Designer zurzeit immer noch nicht genug Gewichtung auf die Benutzbarkeit von Produkten gelegt, wobei dies sich in Zukunft ändern mag.


\section{Bezug auf den Kurs}

Die von uns vorgestellt Prinzipien vom guten Design, könnten in abgewandelter Form in jeden der Beiträge, die i Rahmen der Veranstaltung gehalten wurden, wiedergefunden werden. Dies ist nicht weiter verwunderlich, da es sich dabei um allgemeine Prinzipien handelt, die gebietsübergreifend angewandt werden können. Selbstverständlich müssen je nach Gebiet bestimmte Prinzipien priorisiert werden. Beispielsweise Sichtbarkeit für Blinde eher zu vernachlässigen, dafür Feedback, um so wichtiger.\\
Hingegen gilt "`kenne deinen User"' für alle, was auch in den Vorträgen deutlich wurde, da anfänglich immer die Benutzergruppen mit ihren Fähigkeiten und Bedürfnissen ausgiebig erläutert wurden, was sich auch für den Designprozess als unabdingbar erwies.  \\

Als gemeinsame Schnittmenge konnten wir folgende Richtlinien identifizieren, wobei wir kein Anspruch auf Vollständigkeit erheben:
\begin{itemize}
\item Bei der Gutes konzeptionelles Modell bereitstellen
\item Die Dinge sichtbar machen, darunter fallen auch der Systemzustand sowie die Handlungsmöglichkeiten
\item Natürliche Mappings verwenden, Beziehung zwischen Handlung und Effekt klar darstellen
\end{itemize}

Und ein wichtiger, wenn auch nur indirekt genannter Punkt \textbf{kenne deinen Benutzer!}. 

%Abbildungsverzeichnis

htc mini: \url{http://www.nexustalk.de/wp-content/uploads/2013/01/htc-mini-fernbedienung.jpg}, abgerufen am 24.2.2013
htc butterfly: \url{http://scr.wfcdn.de/8182/HTC-Butterfly-1354876006-0-0.jpg}, abgerufen am 24.2.2013

\begin{thebibliography}{9}
	\bibitem {don} Norman, Donald (1988). The Design of Everyday Things. New York: Basic Books. ISBN 978-0-465-06710-7 
	\bibitem {don2} Norman, Donald (2004). Emotional Design: Why We Love (or Hate) Everyday Things. New York: Basic Books. ISBN 978-0-465-05135-9
	\bibitem {bahn} \url{http://www.bahn.de/p/view/service/vertriebswege/automat/nta.shtml}, abgerufen am 12.2.2013
	\bibitem {ipod} \url{http://storage0.dms.go4it.ro/media/2/84/2022/2305650/13/ipod-classic.jpg}, abgerufen am 14.2.2013
	\bibitem {htc}  \url{http://www.gsmarena.com/htc_launches_mini_a_remote_control_for_htc_butterfly_-news-5416.php}, abgerufen am 21.2.2013
\end{thebibliography}

\end{document}
